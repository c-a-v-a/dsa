\documentclass[11pt]{article}

\usepackage[T1]{fontenc}
\usepackage[polish]{babel}
\usepackage[utf8]{inputenc}
\usepackage{indentfirst}
\usepackage{algpseudocode}
\usepackage{amsmath}

\topmargin=-0.45in
\evensidemargin=0in
\oddsidemargin=0in
\textwidth=6.5in
\textheight=9.0in
\headsep=0.25in

\title{Digital Signature Algorithm}
\author{Filip Cebula, 151410}
\date{\today}

\begin{document}

\maketitle
\pagebreak

\section{Wstęp}
DSA to algorytm kryptograficzny, który służy do generowania i weryfikacji
cyfrowych podpisów. Algorytm ten opiera się na problemie logarytmu
dyskretnego, co sprawia, że jest trudny do złamania. DSA jest
wykorzystywany, do potwierdzania autentyczności i integralności przesyłanych
informacji. Algorytm składa się z trzech części: generacji kluczy, podpisywania
i weryfikacji podpisu.

\section{Generacja kluczy}
Generacje kluczy możemy podzielić na dwie fazy. Pierwszą jest dobór
odpowiednich parametrów algorytmu, a drugą jest generacja klucza prywatnego
i publicznego dla użytkownika.

\subsection{Generacja parametrów}
Generujemy parametry \textbf{(p,q,g)}, przy pomocy których wygenerujemy nasze
klucze.
\begin{enumerate}
  \item Wybieramy parę liczb \textbf{(L,N)}, która oznaczać bedzie rozmiary
    naszych parametrów. Rekomendowane rozmiay to (2048 bitów, 256 bitów).
  \item Wybieramy liczbę \textbf{q}, która jest liczbą pierwszą o \textbf{N}
    bitach długości.
  \item Wybieramy liczbę pierwszą \textbf{p}, taką że liczba \textbf{p-1} jest
    wielokrotnością liczby \textbf{q}.
  \item Wybieramy losowo liczbę całkowitą \textbf{h} z przedziału $[2; p-2]$.
  \item Obliczamy liczbę \textbf{g} ze wzoru $g=h^{(p-1)/q} \pmod p$. Jeżeli
    otrzymamy $g=1$, to wtedy obliczamy g jeszcze raz, wybierając inne
    \textbf{h}.
\end{enumerate}

\subsection{Generacja kluczy}
Generujemy klucz prywatny \textbf{x} i klucz publiczny \textbf{y}, gdzie
\textbf{x} to losowa liczba całkowita z przedziału $[1,q-1]$, a \textbf{y}
obliczamy ze wzoru $y=g^{x} \pmod p$.

\subsection{Przykład}

\subsection{Pseudokod}

\section{Przykładowy program}

\end{document}
